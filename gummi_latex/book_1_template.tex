%%%%%%%%%%%%%%%%%%%%%%%%%%%%%%%%%%%%%%%%%
% Tufte-Style Book (Minimal Template)
% LaTeX Template
% Version 1.0 (5/1/13)
%
% This template has been downloaded from:
% http://www.LaTeXTemplates.com
%
% License:
% CC BY-NC-SA 3.0 (http://creativecommons.org/licenses/by-nc-sa/3.0/)
%
% IMPORTANT NOTE:
% In addition to running BibTeX to compile the reference list from the .bib
% file, you will need to run MakeIndex to compile the index at the end of the
% document.
%
%%%%%%%%%%%%%%%%%%%%%%%%%%%%%%%%%%%%%%%%%

%----------------------------------------------------------------------------------------
%	PACKAGES AND OTHER DOCUMENT CONFIGURATIONS
%----------------------------------------------------------------------------------------

% Use the tufte-book class which in turn uses the tufte-common class
\documentclass{tufte-book}

% Comment this line if you don't wish to have colored links
%\hypersetup{colorlinks}

% Improves character and word spacing
\usepackage{microtype}

% Inserts dummy text
\usepackage{lipsum}

% Better horizontal rules in tables
\usepackage{booktabs}

% Use to insert images into the document
\usepackage{graphicx}
% Sets the default location of images
\graphicspath{{graphics/}}

% Improves figure scaling
\setkeys{Gin}{width=\linewidth,totalheight=\textheight,keepaspectratio}

% Allows customization of verbatim environments
\usepackage{fancyvrb}
% The font size of all verbatim text can be changed here
\fvset{fontsize=\normalsize}

% New command to create parentheses around text in tables which take up no
% horizontal space, this improves column spacing
\newcommand{\hangp}[1]{\makebox[0pt][r]{(}#1\makebox[0pt][l]{)}}
% New command to create asterisks in tables which take up no horizontal
% space, this improves column spacing
\newcommand{\hangstar}{\makebox[0pt][l]{*}}

% Used for printing a trailing space better than using a tilde (~)
% using the \xspace command
\usepackage{xspace}

% A command to print the current month and year
\newcommand{\monthyear}{\ifcase\month
\or January
\or February
\or March
\or April
\or May
\or June
\or July
\or August
\or September
\or October
\or November
\or December\fi\space\number\year}

% Command for printing an epigraph with 2 arguments - the quote and the author
\newcommand{\openepigraph}[2]{
\begin{fullwidth}
\sffamily\large
\begin{doublespace}
\noindent\allcaps{#1}\\ % The quote
\noindent\allcaps{#2} % The author
\end{doublespace}
\end{fullwidth}
}

% Command to insert a blank page
\newcommand{\blankpage}{\newpage\hbox{}\thispagestyle{empty}\newpage}

\usepackage[T1]{fontenc}
\usepackage{beramono}
% Use for listing source code and console content
\usepackage{listings}
\lstset{
  showstringspaces=false,
  formfeed=\newpage,
  tabsize=4,
  aboveskip=10pt,
  belowskip=10pt,
  basicstyle=\ttfamily,
}

\lstdefinestyle{CStyle} {language=C,frame=lines}
\lstdefinestyle{ConsoleStyle} {frame=single}

% Used to generate the index
\usepackage{makeidx}
% Generate the index which is printed at the end of the document
\makeindex

%----------------------------------------------------------------------------------------
%	BOOK META-INFORMATION
%----------------------------------------------------------------------------------------

\title{Book Title}

\author{John Smith}

\publisher{Publisher Name}

%----------------------------------------------------------------------------------------

\begin{document}

\frontmatter

%----------------------------------------------------------------------------------------
%	EPIGRAPH
%----------------------------------------------------------------------------------------

\thispagestyle{empty}
\openepigraph{Quotation 1}{Author, {\itshape Source}}
\vfill
\openepigraph{Quotation 2}{Author}
\vfill
\openepigraph{Quotation 3}{Author}

%----------------------------------------------------------------------------------------

\maketitle % Print the title page

%----------------------------------------------------------------------------------------
%	COPYRIGHT PAGE
%----------------------------------------------------------------------------------------

\newpage
\begin{fullwidth}
~\vfill
\thispagestyle{empty}
\setlength{\parindent}{0pt}
\setlength{\parskip}{\baselineskip}
Copyright \copyright\ \the\year\ \thanklessauthor

\par\smallcaps{Published by \thanklesspublisher}

\par\smallcaps{\url{http://www.bookwebsite.com}}

\par License information.\index{license}

\par\textit{First printing, \monthyear}
\end{fullwidth}

%----------------------------------------------------------------------------------------

\tableofcontents % Print the table of contents

%----------------------------------------------------------------------------------------

\listoffigures % Print a list of figures

%----------------------------------------------------------------------------------------

\listoftables % Print a list of tables

%----------------------------------------------------------------------------------------
%	DEDICATION PAGE
%----------------------------------------------------------------------------------------

\cleardoublepage
~\vfill
\begin{doublespace}
\noindent\fontsize{18}{22}\selectfont\itshape
\nohyphenation
Dedicated to my family and friends.
\end{doublespace}
\vfill
\vfill

%----------------------------------------------------------------------------------------
%	PREFACE
%----------------------------------------------------------------------------------------

\cleardoublepage
\chapter*{Preface} % The asterisk leaves out this chapter from the table of contents

\section{Conventions}
The following typographical conventions are used in this book.

\subsection{Easy to Read Running Text}
The normal text is intentionally typeset in a slightly more narrow width than many
other books. This choice i based on the belief that most readers will prefer this
due to it making the text easy for the eyes to read.

\subsection{Margin Notes}
Margin notes are used extensively throughout the book. The material presented in the 
margins is usually something that is not considered concern the majority of the readers.
Can be background material that most of the readers are already expected to be familiar
with. Can also be special cases that only concerns readers with a special type of 
computer setup. For example a note about something that shall be done if the reader
uses a certain type of operating system. 

\subsection{Terminal Content}
Terminal window content is displayed in a mono spaced font inside
a frame like in the example below. This style is used to display
terminal output as well as for displaying commands that the reader 
shall enter into a terminal console window.

\begin{lstlisting}[style=ConsoleStyle]
sudo apt-get install git
\end{lstlisting} 



%------------------------------------------------

\section{Figures}

\lipsum[1] 

\begin{marginfigure}
\includegraphics[width=\linewidth]{helix}
\caption{This is a margin figure. The helix is defined by 
$x = \cos(2\pi z)$, $y = \sin(2\pi z)$, and $z = [0, 2.7]$. The figure was drawn using
 Asymptote (\url{http://asymptote.sf.net/}).}
\label{fig:marginfig}
\end{marginfigure}

\lipsum[2]

\begin{figure*}[h]
\includegraphics[width=\linewidth]{sine.pdf}
\caption{This graph shows $y = \sin x$ from about $x = [-10, 10]$.
\emph{Notice that this figure takes up the full page width.}}
\label{fig:fullfig}
\end{figure*}

\lipsum[3]

%------------------------------------------------

\section{Tables} \marginnote{This is a random margin note. Notice that there isn't a number preceding the note, and there is no number in the main text where this note was written. Use \texttt{sidenote} to use a number.}

\lipsum[4]

\begin{table} % Add the following just after the closing bracket on this line to specify a position for the table on the page: [h], [t], [b] or [p] - these mean: here, top, bottom and on a separate page, respectively
\centering % Centers the table on the page, comment out to left-justify
\begin{tabular}{l c c c c c} % The final bracket specifies the number of columns in the table along with left and right borders which are specified using vertical bars (|); each column can be left, right or center-justified using l, r or c. To specify a precise width, use p{width}, e.g. p{5cm}
\toprule % Top horizontal line
& \multicolumn{5}{c}{Growth Media} \\ % Amalgamating several columns into one cell is done using the \multicolumn command as seen on this line
\cmidrule(l){2-6} % Horizontal line spanning less than the full width of the table - you can add (r) or (l) just before the opening curly bracket to shorten the rule on the left or right side
Strain & 1 & 2 & 3 & 4 & 5\\ % Column names row
\midrule % In-table horizontal line
GDS1002 & 0.962 & 0.821 & 0.356 & 0.682 & 0.801\\ % Content row 1
NWN652 & 0.981 & 0.891 & 0.527 & 0.574 & 0.984\\ % Content row 2
PPD234 & 0.915 & 0.936 & 0.491 & 0.276 & 0.965\\ % Content row 3
JSB126 & 0.828 & 0.827 & 0.528 & 0.518 & 0.926\\ % Content row 4
JSB724 & 0.916 & 0.933 & 0.482 & 0.644 & 0.937\\ % Content row 5
\midrule % In-table horizontal line
\midrule % In-table horizontal line
Average Rate & 0.920 & 0.882 & 0.477 & 0.539 & 0.923\\ % Summary/total row
\bottomrule % Bottom horizontal line
\end{tabular}
\caption{Table caption text} % Table caption, can be commented out if no caption is required
\label{tab:template} % A label for referencing this table elsewhere, references are used in text as \ref{label}
\end{table}

%----------------------------------------------------------------------------------------

\mainmatter

%----------------------------------------------------------------------------------------
%	CHAPTER 1
%----------------------------------------------------------------------------------------

\chapter{Chapter 1 Title}
\label{ch:1}

%------------------------------------------------

\section{Section 1 - Fullwidth Environment Example}

\begin{fullwidth}
\lipsum[5]
\end{fullwidth}

\subsection{Subsection 1}

\lipsum[6-7]

\subsection{Subsection 2}

\lipsum[7-8]

%------------------------------------------------

\section{Section 2}

\subsection{Subsection 1}

\lipsum[9-10]

\subsection{Subsection 2}

\lipsum[11-12]

%----------------------------------------------------------------------------------------
%	CHAPTER 2
%----------------------------------------------------------------------------------------

\chapter{Chapter 2 Title}
\label{ch:2}

\lipsum[13-20]

%----------------------------------------------------------------------------------------

\backmatter

%----------------------------------------------------------------------------------------
%	BIBLIOGRAPHY
%----------------------------------------------------------------------------------------

\bibliography{bibliography} % Use the bibliography.bib file for the bibliography
\bibliographystyle{plainnat} % Use the plainnat style of referencing

%----------------------------------------------------------------------------------------

\printindex % Print the index at the very end of the document

\end{document}